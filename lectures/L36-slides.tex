\input{configuration}

\title{Lecture 36 --- Course Review }

\author{Jeff Zarnett \\ \small \texttt{jzarnett@uwaterloo.ca}}
\institute{Department of Electrical and Computer Engineering \\
  University of Waterloo}
\date{\today}


\begin{document}

\begin{frame}
  \titlepage

 \end{frame}


\begin{frame}
\frametitle{Exam Coverage}

The exam is based on material we have covered in the lectures \& labs.

The exam is cumulative: covers from the first lecture to the last.

Lab material you are expected to know covers Labs 1 through Lab 4.

\end{frame}

\begin{frame}
\frametitle{Lectures: Introduction}

{\LARGE
Lecture 1: Introduction

Lecture 2: Review of Computer Architecture

Lecture 3: OS Structure and Traps
}

\end{frame}

\begin{frame}
\frametitle{Lectures: Processes}

{\LARGE
Lecture 4: Processes

Lecture 5: Process State

Lecture 6: Processes in UNIX
}

\end{frame}

\begin{frame}
\frametitle{Lectures: SMP}

{\LARGE
Lecture 7: Inter-Process Communication

Lecture 8: Threads

Lecture 9: POSIX Threads

Lecture 10: Symmetric Multiprocessing
}

\end{frame}

\begin{frame}
\frametitle{Lectures: Concurrency}

{\LARGE
Lecture 11: Concurrency \& Atomicity

Lecture 12: Semaphores

Lecture 13: Classical Synchr. Problems
}

\end{frame}

\begin{frame}
\frametitle{Lectures: Deadlock}

{\LARGE
Lecture 15: Deadlock

Lecture 16: Deadlock Avoidance

Lecture 17: Deadlock Detection \& Recovery

Lecture 18: Concurrency/Synchr. in POSIX
}

\end{frame}

\begin{frame}
\frametitle{Lectures: Memory}

{\LARGE
Lecture 19: Memory

Lecture 20: Dynamic Memory Allocation

Lecture 21: Segmentation \& Paging

Lecture 22: Caching

Lecture 23: Virtual Memory

Lecture 24: Virtual Memory II
}

\end{frame}

\begin{frame}
\frametitle{Lectures: Scheduling}

{\LARGE
Lecture 25: Uniprocessor Scheduling

Lecture 26: Scheduling Algorithms

Lecture 27: Scheduling, Idling, Priorities

Lecture 28: Multiproc./Realtime Scheduling

Lecture 29: Scheduling UNIX/Linux/Windows
}

\end{frame}

\begin{frame}
\frametitle{Lectures: I/O, Disk, Files}

{\LARGE
Lecture 30: I/O Devices, Drivers

Lecture 31: Disk Scheduling

Lecture 32: File System Interface

Lecture 33: File System Implementation

Lecture 34: File Allocation Methods
}

\end{frame}

\begin{frame}
\frametitle{Lectures: Virtualization}

{\LARGE
Lecture 35: Virtualization
}

\end{frame}

\begin{frame}
\frametitle{Basic Exam Information}

\begin{table}[h]
 \begin{tabular}{|l l|}
        	\hline
			~ & ~ \\	
			Date of Exam: & \textbf{Tuesday, 15 December 2015} \\
			Time Period: & \textbf{Start: 16:00. End: 18:30}\\
			Duration of Exam: & \textbf{150 minutes}\\
			Exam Type: & \textbf{Closed Book}\\
			Additional Materials: & \textbf{Non-Programmable Calculators}\\
			~ & ~\\
			\hline
          \end{tabular}
\end{table}

We are writing in: MC\\
Check Odyssey for your assigned seat.

\end{frame}

\begin{frame}
\frametitle{Exam Instructions}

\begin{enumerate}
	\item No aids are permitted except for non-programmable calculators.
	\item Turn off all communication devices.
	\item There are four (4) questions, some with multiple parts. Not all are equally difficult.
	\item The exam lasts 150 minutes and there are 100 marks.
	\item If you feel like you need to ask a question, know that the most likely answer is ``Read the Question''. No questions are permitted. If you find that a question requires clarification, proceed by clearly stating any reasonable assumptions necessary to complete the question. If your assumptions are reasonable, they will be taken into account during grading. 
\end{enumerate}

\end{frame}

\begin{frame}
\frametitle{Exam Stats}

Preliminary statistics on the exam:

\begin{itemize}
	\item About 70\% of the exam is post-midterm material
	\item About 20\% of the exam is C programming
\end{itemize}

\end{frame}

\begin{frame}
\frametitle{How to Prepare}

How to prepare for the final exam:

\begin{enumerate}
	\item Review lecture notes and slides.
	\item Review the tutorial slides.
	\item Understand your lab solutions.
	\item Try old exams.
	\item Ask for extra help if you need it (we have many TAs + 1 instructor).
\end{enumerate}

\end{frame}

\begin{frame}
\frametitle{Exam Tips}

Tips for the Exam:

\begin{enumerate}
	\item Take the time to read the question carefully.
	\item You can use point form instead of full sentences.
	\item Don't leave questions blank - Nothing on the page = 0 marks.
	\item Do the questions you know (or find easy) first, then move on to more challenging ones.
	\item Keep an eye on the time.
	\item Sleep the night before (all nighters are bad).
	
\end{enumerate}

\end{frame}

\begin{frame}
\frametitle{About Grades}

No grades can be released until after the end of exams.

You'll see them in Quest when grades for the term become available.


\end{frame}


\end{document}

