\input{configuration}

\title{Lecture 19 --- Memory }

\author{Jeff Zarnett \\ \small \texttt{jzarnett@uwaterloo.ca}}
\institute{Department of Electrical and Computer Engineering \\
  University of Waterloo}
\date{\today}


\begin{document}

\begin{frame}
  \titlepage

 \end{frame}


\begin{frame}
\frametitle{Main Memory}

Execution: the CPU fetches instructions from memory.

Maybe it requires fetching of operands, also from memory. 

A result may be stored back in memory. 

A single simple instruction like an addition could result in 4 memory accesses. 

Therefore, we are spending a lot of time interacting with memory.


\end{frame}



\begin{frame}
\frametitle{}


\end{frame}

\begin{frame}
\frametitle{}


\end{frame}

\begin{frame}
\frametitle{}


\end{frame}

\begin{frame}
\frametitle{}


\end{frame}

\begin{frame}
\frametitle{}


\end{frame}

\begin{frame}
\frametitle{}


\end{frame}

\begin{frame}
\frametitle{}


\end{frame}

\begin{frame}
\frametitle{}


\end{frame}

\begin{frame}
\frametitle{}


\end{frame}

\begin{frame}
\frametitle{}


\end{frame}

\begin{frame}
\frametitle{}


\end{frame}

\begin{frame}
\frametitle{}


\end{frame}

\begin{frame}
\frametitle{}


\end{frame}

\begin{frame}
\frametitle{}


\end{frame}

\begin{frame}
\frametitle{}


\end{frame}

\begin{frame}
\frametitle{}


\end{frame}

\begin{frame}
\frametitle{}


\end{frame}

\begin{frame}
\frametitle{}


\end{frame}
\end{document}

